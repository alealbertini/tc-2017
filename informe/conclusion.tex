\section{Conclusiones}
Luego de analizar 3 redes con configuraciones diferentes podemos concluir que el trafico unicast parece 
ser usualmente superior al broadcast si se monitorea una red durante el tiempo suficiente, tal vez por que las 3 redes tienen un proposito
comun que es comunicar una red local al exterior u otra red. Tambi'en podemos mencionar que nuestra elecci'on de nodos distinguidos result'o
ser muy 'util, no solo para distinguir el usual gateway por default si no como vimos tambien para detectar otros gateways no asignados al
nodo monitor o bien para diagnosticar nodos que no funcionan bien como vimos en el primer caso. Esta elecci'on se vali'o exclusivamente en la
entrop'ia de la fuente la cual sirvi'o efectivamente como un limite al cual un simbolo/nodo debe acercarse para ser relevante en la red y que la
entropi'a parece ser mayor cuando mayor cantidad de nodos hay en la red, esto tiene sentido si se modela usando ARP dado que vamos a incrementar
la cantidad de simbolos con poca frencuencia en nuestra fuente.
Cabe destacar tambi'en el descubrimiento de tecnicas utilizadas por routers y nodos para comunicar datos a traves de ARP con el objetivo
de mejorar la performance de la red, utilizando mensajes no requeridos por ning'un otro nodo.


