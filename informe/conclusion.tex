\section{Conclusiones}
Luego de analizar cuatro redes con configuraciones diferentes podemos concluir que el tr'afico unicast parece 
ser usualmente superior al broadcast si se monitorea una red durante el tiempo suficiente, tal vez por que las cuatro redes tienen un proposito
com'un que es comunicar una red local al exterior u a otra red. Adem'as la frecuencia de los paquetes broadcast aumenta considerablemente
 para el caso de la red p'ublica, ya que es m'as probable que se quiera transmitir un mismo paquete para todos los hosts. \\
 
 Tambi'en podemos mencionar que nuestra elecci'on de nodos distinguidos result'o
ser muy 'util, no solo para distinguir el usual gateway por default sino como vimos tambi'en para detectar otros gateways no asignados al
nodo monitor o bien para diagnosticar nodos que no funcionan bien, como vimos en el primer caso. Esta elecci'on se vali'o exclusivamente en la
entrop'ia de la fuente, la cual sirvi'o efectivamente como un l'imite al cual un s'imbolo/nodo debe acercarse para ser relevante en la red y que la
entropi'a parece ser mayor cuando mas cantidad de nodos hay en la red, esto tiene sentido si se modela usando ARP dado que vamos a incrementar
la cantidad de s'imbolos con poca frencuencia en nuestra fuente. \\

Cabe destacar tambi'en el descubrimiento de t'ecnicas utilizadas por routers y nodos para comunicar datos a trav'es de ARP con el objetivo
de mejorar la performance de la red, utilizando mensajes no requeridos por ning'un otro nodo.


