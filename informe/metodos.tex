\section{M'etodos y ambientes}
Las redes ser'an analizadas con dos herramientas que escuchan un medio compartido en modo promiscuo. \\

La primera herramienta
 modela la red como una fuente $S$ cuyos posibles s'imbolos son $S_{unicast}$ si el paquete le'ido es enviado
 a un nodo espec'ifico y $S_{broadcast}$ si el paquete es enviado a la direcci'on broadcast (\texttt{ff:ff:ff:ff:ff:ff}). 
 Adem'as, ning'un paquete es filtrado y cualquier paquete que llega nuestro enlace ser'a tenido en cuenta.
 El objetivo en este caso es diferenciar cuando se env'ia un paquete a un s'olo dispositivo (paquete unicast) o cuando se manda
 un mismo paquete a todos los dispositivos que est'an conectados a la red en ese momento (paquete broadcast).\\
 
La segunda herramienta modelar'a la red como una fuente $S1$ cuyos posibles s'imbolos son todas las direcciones IP disponibles
en la red (como esto es desconocido, vamos a asumir que todos los s'imbolos observados son todos los s'imbolos posibles) en 
paquetes ARP que se van capturando en la red. Adem'as, se agrupar'an los nodos iguales para mostrar un gr'afico m'as entendible. 
El objetivo de 'esta fuente es poder distinguir los distintos hosts que hay en la red, poder analizar como est'an conectados entre s'i, 
 descubrir el alcance total de la red y ver que dispositivos est'an conectados en la misma.\\ 

La consigna era definir una funci'on que designe alg'un s'imbolo como distinguido
fundado en alg'un resultado matem'atico. Por eso, distinguimos a aquellos s'imbolos que tengan informaci'on menor a
la entrop'ia, dado que estos ser'an m'as comunes de ver (por la definici'on de informaci'on y de entrop'ia), ya que los otros
nodos de la red piden su direcci'on de enlace muy frecuentemente. Esto har'a distinguir a nodos internos muy usados o bien
al gateway por defecto de la red, por la cual, todos los nodos salen a otras redes como internet.\\

El tiempo de muestreo es similar para todos los experimentos, fue de 30 minutos en cada caso. \\

Las 4 redes que elegimos para experimentar son las siguientes:
\begin{itemize}
	\item Red hogare\~na: red local wi-fi en una casa particular, lo 'unico que sabemos es que contiene una cantidad muy
	 limitada de computadoras personales y dispositivos m'oviles.
	\item Red laboral: red local ethernet en una oficina, sabemos que conviven computadoras de empleados y servidores.
	\item Red p'ublica: red wi-fi en un shopping, diferentes tipos de usuarios, dispositivos y duraci'on de leasing the direcciones de red
	\item Red hogare\~na nuevo experimento: Para el recupetario del TP1, se decidi'o agregar 'esta nueva red de hogar, mucho m'as 
		peque\~na que la primera red hogare\~na y en un domicilio diferente. 
\end{itemize}
