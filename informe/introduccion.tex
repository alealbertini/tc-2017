\section{Introducci'on}
En este informe nos proponemos analizar diferentes redes locales utilizando herramientas derivadas de la
teor'ia de la informaci'on. El objetivo es usar y analizar datos de los paquetes que pasan
por la red para descubrir patrones y topologias sin conocer previamente nada sobre los nodos de la red. Para lograr esto
utilizamos direcciones de la capa de enlace en un principio y luego datos de paquetes \texttt{ARP} (address resolution protocol,
un protocolo  utilizado para obtener una direcci'on de enlace dada una direcci'on de red para que dos nodos en 
una misma red local de acceso compartido puedan comunicarse) para obtener mediciones de informaci'on y entropia de la misma
red modelada de modo diferente. Adem'as, se realizar'an distintos tipos de gr'aficos, con las mediciones obtenidas y las fuentes propuestas,
para tratar de mostrar y detallar los resultados obtenidos. Se mostrar'an distintos tipos de 
gr'aficos, como la composici'on de una red y sus nodos o la cantidad de paquetes broadcast vs unicast,
y se analizar'a el por qu'e de esos resultados obtenidos.


