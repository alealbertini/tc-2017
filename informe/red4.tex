\section{Red hogare\~na casa Ale}

En 'esta secci'on se analizar'a una red hogare\~na pero del hogar de otro integrante. El objetivo en este caso, es mostrar una red mucho m'as chica,
con un grafo con menor cantidad de nodos (hosts) y ver si la cantidad de paquetes Unicast y Broadcast son similares o no. 
Para este caso se espera que los valores en promedio de paquetes Unicast y Broadcast sean parecidos al experimento realizado en el hogar del 
otro integrante (punto 1) red hogare\~na), ya que en un hogar, cada dispositivo m'ovil 
y cada computadora realizan sus propias tareas, los usuarios de cada dispositivo realizan sus propias actividades y por lo general, no est'an 
relacionadas entre s'i, por lo tanto, la cantidad de paquetes Unicast deber'ia ser mayor, claramente, a la cantidad de paquetes Broadcast. \\

Dada la fuente binaria S, donde S se define como S = ($S_{unicast}$, $S_{broadcast}$), tomando como la entrop'ia m'axima = 1, suponiendo 
que los eventos son equiprobables, a continuaci'on se mostrar'an los calculos de la entrop'ia y los porcentajes de paquetes Unicast vs Broadcast:


\begin{figure}[!h]
\centering
\caption{Informaci'on de S p'ublica}
\begin{tabular}{ r|c|c| }
\multicolumn{1}{r}{}
 &  \multicolumn{1}{c}{frecuencia}
 & \multicolumn{1}{c}{informaci'on} \\
\cline{2-3}
$S_{broadcast}$ & 0 & 0 \\
\cline{2-3}
$S_{unicast}$ & 0 & 0 \\
\cline{2-3}
\end{tabular}
\end{figure}

Como se puede observar en la siguiente tabla, hay dos nodos que tienen valores que est'an por debajo de la entrop'ia media,
el host 158.124.30.40 y el host 158.124.30.100. El primero se debe a la computadora de escritorio que est'a conectada al router y el 
segundo se debe a uno de los celulares que m'as actividad en internet estaba teniendo al momento de la captura de paquetes. El resto
de los nodos de deben al resto de los dispositivos m'oviles. \\ 

A continuaci'on se podr'a obsevar la actividad de cada nodo, en cuanto a cantidad de informaci'on con respecto a la fuente S1,
que seg'un como se defini'o anteriormente, es la cantidad de nodos distintos que aparecen en la red. Estos datos se comparan 
contra la entrop'ia media que es la l'inea roja que aparece en el gr'afico. 
 
\begin{figure}[!h]
\centering
\caption{Informaci'on red p'ublica}
\includegraphics[width=0.75\textwidth]{red4_info}
 \label{fig:red3info}
\end{figure}

En el siguiente gr'afico se mostrar'a el grafo subyacente de mensajes ARP que se form'o con los paquetes ARP que estaban circulando 
en la red que observamos. Como se descubri'o antes, aparecen en color amarillo, los nodos destacados, es decir, que sus valores de
 informaci'on est'an por debajo de la entrop'ia media. Se pueden observar cuatro nodos, un nodo destacado que es el principal (el 158.124.30.40),
 esa ip se corresponde con la computadora de escritorio que esta conectada por cable de red al router, este puede ser un motivo por el cual,
 los nodos restantes est'en enlazados a este, en el grafo. Los nodos restantes se corresponden a telefonos m'oviles, esto se pudo averiguar 
 chequeando la ip de cada celular. Por lo tanto, en este caso, no se detectaron nodos an'omalos. \\
 

\begin{figure}[!h]
\caption{Visualizaci'on red p'ublica, en amarillo los nodos destacados}
\includegraphics[width=1.1\textwidth]{red4_red}
 \label{fig:red3net}
\end{figure}

