\section{Red hogare\~na}
Esta red en principio es muy controlada y s'olo sabemos que cuenta con un n'umero peque\~no de nodos. La red usa
en su mayoria wi-fi por lo cual es de esperar que el nodo que usamos para monitorear la red reciba amplia cantidad de
paquetes who-has y is-at dado que no hay switches segmentando la red que filtren respuestas is-at de terceros.

El primer experimento, el cual ve la red como una fuente con 2 s'imbolos unicamente, arroja luego de monitorear la red,
que la mayor parte del tr'afico es unicast y solo el 17\% es broadcast.\\

\begin{figure}[!h]
\centering
\caption{Informaci'on de S, Red hogare\~na}
\begin{tabular}{ r|c|c| }
\multicolumn{1}{r}{}
 &  \multicolumn{1}{c}{frecuencia}
 & \multicolumn{1}{c}{informaci'on} \\
\cline{2-3}
$S_{broadcast}$ & 0.17 & 2.56 \\
\cline{2-3}
$S_{unicast}$ & 0.83 & 0.27 \\
\cline{2-3}
\end{tabular}
\end{figure}
 
Estos valores nos dan una entrop'ia de 0.66 bits (siendo el m'aximo 1 dado que hay 2 simbolos en principio equiprobables) lo
 cual concuerda con las expectativas dado que en una red local hogare\~na se espera ver mas que nada trafico unicast entre
 los nodos y el gateway hacia internet.\\
 
En el segundo experimento, el cual modela la red basado en la direcci'on a resolver en mensajes ARP, usamos la misma captura
utilizada en el experimento anterior. El script que usamos para analizar los resultados destac'o 3 nodos de la red cuando
esperabamos s'olo 1. El siguiente gr'afico muestra la informaci'on de cada s'imbolo y la entrop'ia de la fuente, como se puede
ver hay 3 nodos destacados bajo nuestra definici'on.\\
 
\begin{figure}[!h]
\centering
\caption{Informaci'on red hogare\~na}
\includegraphics[width=0.55\textwidth]{red1_info}
 \label{fig:red1info}
\end{figure}

Una investigaci'on m'as meticulosa concluy'o que \texttt{192.168.1.1} es el gateway por defecto de la red mientras que los otros 2 nodos
destacados, \texttt{192.168.1.11} y \texttt{192.168.1.101} ni siquiera existen en la red, las direcciones no fueron asignadas y ning'un
nodo las est'a usando. S'olo podemos
suponer que este host tiene un software mal configurado o defectuoso.
Las interacciones de los nodos se pueden ver en m'as detalle, en el siguiente gr'afico donde se puede ver a los nodos destacados
y las relaciones con los otros nodos. Notar que en el gr'afico, as'i como en el experimento, tambi'en aparece un nodo con direcci'on 
\texttt{169.254.255.255} la cual es la direcci'on que el sistema operativo windows asign'o a un nodo cuando este no puede
contactar a ning'un servidor DHCP para que le asigne una direcci'on libre. Usualmente los nodos operan con esta direcci'on durante
segundos o minutos, hasta que se pueda proveer de una direcci'on v'alida.

\begin{figure}[!h]
\centering
\caption{Visualizaci'on red hogare\~na, en amarillo los nodos destacados}
\includegraphics[width=0.9\textwidth]{red1_red}
 \label{fig:red1net}
\end{figure}

El gr'afico confirma que los nodos que se destacaron, son solo accedidos por un nodo.\\
